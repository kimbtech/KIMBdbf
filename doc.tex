%
%	Allgemein
%
\documentclass[paper=A4,pagesize=auto,12pt,headinclude=true,footinclude=true,BCOR=0mm,DIV=calc]{scrartcl}

\usepackage[head=30pt,foot=30pt]{geometry}

\renewcommand*{\familydefault}{\sfdefault}
\usepackage[T1]{fontenc}
\usepackage{lmodern}
\usepackage[utf8]{inputenc}

\usepackage[ngerman]{babel} % neue deutsche Trennungsregeln, etc

\usepackage[scaled=1]{helvet} %Schriftart

% Schönere Schriftart für nicht proportionale Schrift laden
\usepackage{courier}

\usepackage{graphicx}
\usepackage{amssymb, amsfonts, amsthm, amsmath, float}

\usepackage{setspace}   
\onehalfspacing % 1,5 Zeilenabstand

%
%	Projektinformationen
%
\newcommand{\allgclassname}{KIMBdbf}
\newcommand{\allgclassversion}{4.5}

%
%	PDF Meta
%
\usepackage[
	pdftitle={\allgclassname},
	pdfsubject={Dokumentation für die KIMB-technologies
	 \allgclassname ~PHP-Klasse},
	pdfauthor={KIMB-technologies},
	pdfcreator={KIMB-technologies PDF},
	pdfproducer={KIMB-technologies PDF}
]{hyperref}
\pdfcompresslevel=9
\pdfobjcompresslevel=9

%
%	Quellcode
%
\usepackage{listings}
\usepackage{xcolor}
\usepackage{textcomp}
\lstset{%
  basicstyle=\ttfamily,%
  showstringspaces=false,%
  upquote=true}
\lstdefinestyle{pseudo}{language={},%
  basicstyle=\normalfont,%
  morecomment=[l]{//},%
  morekeywords={for,to,while,do,if,then,else},%
  mathescape=true,%
  columns=fullflexible}
% \begin{lstlisting}[gobble=4,language=Java]
%     public class Main {
%       public static void main(String[] args) {
%         System.out.println("Hallo Welt");
%       }
%     }
%  \end{lstlisting}

%
%	Fußzeile und Kopfzeile
%
\usepackage{fancyhdr} 
\pagestyle{fancy} 
\lhead{\allgclassname}
\chead{Version \allgclassversion}
\rhead{\thepage} 
\lfoot{\today}
\cfoot{}
\rfoot{\copyright ~KIMB-technologies}
\renewcommand{\footrulewidth}{0.2pt}
\renewcommand{\headrulewidth}{0.2pt}

%
%	Links ohne Box
%
\hypersetup{
    colorlinks,
    linkcolor={black},
    citecolor={black},
    urlcolor={blue}
}



\begin{document}

  %Titelseite
  \title{Dokumentation für die \allgclassname ~PHP-Klasse}
  \author{KIMB-technologies}
  \date{\today}
  \maketitle
  
  \thispagestyle{empty}
  
  %Inhaltsverzeichnis
  \newpage
  \tableofcontents

  %
  %Seitenteile:
  %

  \newpage
  \section{Allgemein}
  
  Die KIMBdbf PHP-Klasse (KIMB databasefile) für PHP ersetzt eine Datenbank, sie kann
  aber auch in ergänzug zu einer Datenbank verwendet werden, z.B. zur Speicherung der 
  Zugangsdaten.\\
  Alle Datensätze werden komplett in Dateien direkt auf dem Server gespeichert.
  Die Klasse ist komplett in PHP geschrieben und jedes Skript muss nur die Klassendatei laden
  per \lstinline-require- oder mit einem autoload.\\
  Die Klasse beinhaltet viele Methoden zur Datenspeicherung, so ist lesen, schreiben und Suchen von bestimmten
  Datensätzen möglich. Es Funktionen die den elementaren Methoden entsprechen.\\
  Es wird PHP 5 (größer 5.5.0) oder PHP 7 benötigt. Für verschlüsselte Dateien ist PHP Mcrypt erforderlich.\\
  
  \subsection{Dateiaufbau}
  \subsubsection{Beispieldatei}
  
  \begin{lstlisting}[gobble=2,language=Java]
   
  <[about:doc]>KIMB dbf V4.10 - KIMB-technologies<[about:doc]>
  <[user]>mmuster<[user]>
  <[name]>Max Mustermann<[name]>
  <[aboutme]>Ich bin der Max.<[aboutme]>
  <[passwort]>2f4cdd81258da5f104bb343ab9d13ef<[passwort]>
  <[groups]>chemie<[groups]>
  <[groups]>physik<[groups]>
  <[membership1]>newsletter chemie<[membership1]>
  <[membership2]>newsletter physik<[membership2]>
  <[membership3]>newsletter uni<[membership3]>
  <[105]>note==abschluesse<[105]>
  <[105-note]>1<[105-note]>
  <[105-abschluesse]>Chemie<[105-abschluesse]>
  
  \end{lstlisting}
  
  \subsubsection{Datensatzarten}
  
  Es gibt drei verschiedenen Arten von Datensätzen:
  
  \begin{description}
    \item[Normal] \hfill \\
      Es wird einem Tag ein Inhalt zugeordnet.\\
      Tags können einmal oder mehrfach vorhanden sein.\\
      \begin{itshape}
	Ein Tag kann nur ersetzt oder gelöscht werden, wenn dieser nur einmal
	in einer Datei vorhanden ist.\\
	Es kann aber in den Inhalten mehrfach vorhandener
	Tags gesucht werden.
      \end{itshape}
      \\
      \textbf{Beispielanwendung:} Logdatei, Benutzerdatei, Konfigurationsdatei
    \item[Aufzählung] \hfill \\
      Es werden einem Tag mehrere Inhalte zugeordnet.\\
      \begin{itshape}
	Es können einzelne Inhalte gelöscht werden.
      \end{itshape}
      \\
      \textbf{Beispielanwendung:} Zugriffsrechte, Newsletter, Gruppen
    \item[ID-Zuordnung] \hfill \\
      Es wird eine ID erstellt, welche eigenen Untergruppen
      erhält und diese mit Inhalten füllen kann.\\
      \begin{itshape}
	Es können alle Inhalte einer ID bestimmt werden,
	genauso können alle IDs nach einem bestimmten Inhalt
	durchsucht werden.\\
	Es können einzelne Inhalte bearbeitet werden.\\
	Genau wie eine Tabelle einer Datenbank.
      \end{itshape}
      \\
      \textbf{Beispielanwendung:} Auflistungen mit Attributen, Benutzerdatei, Konfigurationsdatei\\
      Alle IDs in einer Datei werden automatisch der Aufzählung \lstinline* allidslist * hinzugefügt.
  \end{description}

  
  \subsection{Lizenz}
  Die KIMBMySQL PHP-Klasse ist unter der GPLv3 veröffentlicht.\\
  \href{http://www.gnu.org/licenses/gpl-3.0.txt}{http://www.gnu.org/licenses/gpl-3.0.txt}

  \section{Einbau}
  
  Die KIMBdbf PHP-Klasse kann per autoload in ein Projekt eingebunden werden. Sie ist unter dem \lstinline-namespace KIMBdbf;- zu erreichen.\\
  Die Klassendatei erhalten Sie (\href{https://github.com/kimbtech/KIMBdbf}{hier}).\\
  \\
  Zur Speicherung der Daten werden Dateien genutzt. Ohne besondere Konfiguration wird automatisch
  ein Unterordner „/kimb-data/“ bei der Klassendatei zur Speicherung der Dateien gewählt.
  (Dieser kann bei der Initialisierung des Objektes anders gewählt werden.)\\
  Dieser Ordner muss vorhanden und für PHP schreibbar sein, aus Sicherheitsgründen sollte er nicht mit einem Browser zu erreichen sein.\\
  \\
  Seit Version 4.5 werden die KIMB-Dateien beim Schreiben durch die KIMBdbf-Klasse gesperrt, es ist somit
  kein paralleles Schreiben verschiedener Prozesse/ Objekte möglich.\\
  Sollte das Schreiben einmal nicht möglich sein, führt die KIMBdbf 10 Versuche durch,
  doch noch Schreibrechte zu erhalten (mit je 0,75 Sekunden Pause). Sollte sie dabei keine Rechte
  erhalten, wird das Skript beendet!\\
  Gleichzeitiges Lesen in einer Datei von verschiedenen Prozessen/ Objekten aus stellt kein Problem dar!\\
    
  \newpage
  \section{Verwendung}
  
  \subsection{Kodierung, Vergebene Zeichen}
  
  Die KIMBdbf verlangt keine bestimmte Kodierung, da PHP Datensätze immer so in Dateien
  speichert, wie es diese erhält. Das bedeutet, man muss sich für eine Kodierung entschieden
  und alle Datensätze kommen dann so zurück wie diese übergeben wurden.\\
  Es ist sicherlich sinnvoll weit verbeitete Kodierungen zu verenden, z.B. UTF-8.\\
  \\
  \begin{description}
   \item[Dateiname] \hfill \\
    Dateinamen dürfen nur aus Zeichen von A-Z (groß und klein) sowie Ziffern bestehen. Außderdem sind
    die folgenden Sonderzeichen erlaubt: \lstinline* _ . - / *\\
    Deutsche Umlaute, das Sz sowie Punkte und Leerzeichen, werden automatisch umgeschrieben. Alle
    anderen verbotenen Zeichen werden herausgeschnitten.
   \item[Tag] \hfill \\
    Die Tags dürfen aus allen Zeichen bestehen, mit Ausnahmen von \lstinline* <[ * , \lstinline* ]> * und \lstinline* about:doc *,
    welche zu \lstinline* < * , \lstinline* > * und \lstinline* aboutdoc * umgeschrieben werden.\\
    Auch Zeilenumbrüche sowie Tabulatoren sind verboten, diese werden herausgeschnitten.
   \item[Inhalt] \hfill \\
    Inhalte können alle Zeichen haben. \\
    \\
    \begin{itshape}
     Intern werden einige Zeichen bzw. Zeichenketten kodiert gespeichert und bei der Ausgabe zurückverwandelt.
     (Umbrüche, Tabulatoren, \lstinline* <[ * , \lstinline* ]> * und \lstinline* == *)
     Die kodierte Speicherung benötigt etwas mehr Speicherplatz.
    \end{itshape}
    \\
    \\
    Ab Version 4.5 ist es möglich neben Strings als Inhalten auch Arrays as Inhalte zu speichern. \\
    \begin{itshape}
      Die Arrays werden bei der Speicherung automatisch erkannt und als JSON abgelegt.\\
      Für die Ausgabe wird das JSON dann wieder zu PHP-Arrays konvertiert.\\
      \\
      In der Dokumentation ist weiterhin String als Typ für die Parameter und Rückgaben der Inhalte angeben.
      Sie können hier aber auch Arrays übergeben, bzw. müssen mit Rückgaben als Array rechnen.\\
      (Arrays kommen natürlich nur zurück, wenn diese auch bei der Speicherung übegeben wurden.)
    \end{itshape}

  \end{description}

  \newpage
  
  \subsection{Objekt erstellen}
  
   Erstellen Sie zuerst ein Objekt der KIMBdbf PHP-Klasse.
   \begin{lstlisting}[gobble=4,language=Java]
     $dbf = new KIMBdbf\KIMBdbf(
	      $datei [, $encryptkey = 'off' [, $path = __DIR__ ]]
	    );
   \end{lstlisting}
   
   \begin{tabular}{|lcp{0.7\textwidth}|}
 	  \hline
 	    \multicolumn{3}{|l|}{ \textbf{Parameter} } \\
 	  \hline
 	    \$datei & String & \begin{itshape} Name der KIMB-Datei, in der das Objekt arbeiten soll. (üblicherweise *.kimb) (Unterordner möglich) (Datei muss nicht vorhanden sein, wird automatisch erstellt) \end{itshape} \\
 	  &&\\
 	    \$encryptkey & Sting & \begin{itshape} Passwort, mit dem die Datei verschlüsselt wurde/ verschlüsselt werden soll. (Verschlüsselung über PHP Mcrypt; \lstinline* MCRYPT_BLOWFISH , MCRYPT_MODE_CBC * ) (nicht übergeben => \lstinline* off * => keine Verschlüsselung) \end{itshape} \\
 	  &&\\
 	    \$path & String & \begin{itshape} Pfad zum Ordner „/kimb-data/“. (nicht übergeben => \lstinline* __DIR__ * => Unterordner direkt bei der Klassendatei) \end{itshape} \\
 	  \hline
 	    \multicolumn{3}{|l|}{ \textbf{Rückgabe} } \\
 	  \hline
 	    \multicolumn{3}{|c|}{ KIMBdbf Objekt } \\
 	  \hline
  \end{tabular}
  Anschließend können sie gleich mit dem Objekt arbeiten. Sie können natürlich auch mehrere KIMBdbf Objekte Parallel verwenden.

  \subsection{Methodenübersicht}
  Die Übersicht zeigt die verschiedenen Methoden, sortiert nach den Datensatzarten.
   
  \begin{tabular}{|l|p{0.4\textwidth}|p{0.4\textwidth}|}
      \hline
	& \textbf{ Lesen \& Suchen } & \textbf{ Schreiben } \\
      \hline
	\textbf{ Normal } & \hyperref[sec:mth_one_read]{ \$this->read\_kimb\_one } & \hyperref[sec:mth_new]{ \$this->write\_kimb\_new } \\
			  & \hyperref[sec:mth_all]{ \$this->read\_kimb\_all } & \hyperref[sec:mth_replace]{ \$this->write\_kimb\_replace } \\
			  & \hyperref[sec:mth_search]{ \$this->read\_kimb\_search } & \hyperref[sec:mth_del]{ \$this->write\_kimb\_delete } \\
			  &							& \hyperref[sec:mth_one_write]{ \$this->write\_kimb\_one } \\
      \hline
	\textbf{ Aufzählung } & \hyperref[sec:mth_all_tpl]{ \$this->read\_kimb\_all\_teilpl } & \hyperref[sec:mth_tpl]{ \$this->write\_kimb\_teilpl } \\
			      & \hyperref[sec:mth_search_tpl]{ \$this->read\_kimb\_search\_teilpl } & \hyperref[sec:mth_tpl_del]{ \$this->write\_kimb\_teilpl\_del\_all } \\
      \hline
	\textbf{ ID-Zuordnung } & \hyperref[sec:mth_id]{ \$this->read\_kimb\_id } & \hyperref[sec:mth_id_next]{ \$this->next\_kimb\_id } \\
				& \hyperref[sec:mth_id_all_xx]{ \$this->read\_kimb\_all\_xxxid } & \hyperref[sec:mth_id_write]{ \$this->write\_kimb\_id } \\
				& \hyperref[sec:mth_id_all]{ \$this->read\_kimb\_id\_all } & \hyperref[sec:mth_id_write_arr]{ \$this->write\_kimb\_id\_array } \\
				& \hyperref[sec:mth_id_search]{ \$this->search\_kimb\_id } & \\
				& \hyperref[sec:mth_search_xx]{ \$this->search\_kimb\_xxxid } & \\
      \hline
      \hline
	\textbf{ Weitere } & \multicolumn{2}{|c|}{ \hyperref[sec:mth_fshow]{ \$this->show\_kimb\_file } } \\
			  & \multicolumn{2}{|c|}{ \hyperref[sec:mth_fdel]{ \$this->delete\_kimb\_file } } \\
      \hline
  \end{tabular}
  \label{tab:methodenuerbersicht}
  \textit { Klicken Sie auf eine Methode und sie werden zur genauen Erklärung geleitet. }
%
%
%
  \subsection{Methoden} 
  \subsubsection{\$this->read\_kimb\_one}
  \label{sec:mth_one_read}
	    Lesen des Inhalts eines Tags. (wenn mehrfach vorhanden, erster Inhalt)
	    \begin{lstlisting}[gobble=4,language=Java]
	      $dbf->read_kimb_one( $teil );
	    \end{lstlisting}
	    
	    \begin{tabular}{|lcp{0.7\textwidth}|}
		    \hline
		      \multicolumn{3}{|l|}{ \textbf{Parameter} } \\
		    \hline
		      \$teil & String & \begin{itshape} Tag des Inhalts \end{itshape} \\
		    \hline
		      \multicolumn{3}{|l|}{ \textbf{Rückgabe} } \\
		    \hline
			     & String & \begin{itshape} Inhalt \end{itshape} \\
		    \hline
	    \end{tabular}
	    \begin{flushright} \small \hyperref[tab:methodenuerbersicht]{Zur Methodenübersicht} \end{flushright}
  
  \subsubsection{\$this->read\_kimb\_all}
  \label{sec:mth_all}
	    Lesen aller Inhalte eines Tags.
	    \begin{lstlisting}[gobble=4,language=Java]
	      $dbf->read_kimb_all( $teil );
	    \end{lstlisting}
	    
	    \begin{tabular}{|lcp{0.7\textwidth}|}
		    \hline
		      \multicolumn{3}{|l|}{ \textbf{Parameter} } \\
		    \hline
		      \$teil & String & \begin{itshape} Tag der Inhalte \end{itshape} \\
		    \hline
		      \multicolumn{3}{|l|}{ \textbf{Rückgabe} } \\
		    \hline
			     & Array & \begin{itshape} Inhalte \end{itshape} \\
		    \hline
	    \end{tabular}
	    \begin{flushright} \small \hyperref[tab:methodenuerbersicht]{Zur Methodenübersicht} \end{flushright}
  
  \subsubsection{\$this->read\_kimb\_search}
  \label{sec:mth_search}
	   Suchen nach einem Tag mit bestimmtem Inhalt, bei mehrfach vorhanden Tags.
	    \begin{lstlisting}[gobble=4,language=Java]
	      $dbf->read_kimb_search( $teil, $search );
	    \end{lstlisting}
	    
	    \begin{tabular}{|lcp{0.7\textwidth}|}
		    \hline
		      \multicolumn{3}{|l|}{ \textbf{Parameter} } \\
		    \hline
		      \$teil & String & \begin{itshape} Tag der zu durchsuchenden Inhalte \end{itshape} \\
		      \$search & String & \begin{itshape} Gesuchter Inhalt \end{itshape} \\
		    \hline
		      \multicolumn{3}{|l|}{ \textbf{Rückgabe} } \\
		    \hline
			     & Boolean & \begin{itshape} Gefunden? \end{itshape} \\
		    \hline
	    \end{tabular}
	    \begin{flushright} \small \hyperref[tab:methodenuerbersicht]{Zur Methodenübersicht} \end{flushright}
  
  \subsubsection{\$this->write\_kimb\_new}
  \label{sec:mth_new}
	    Einen Inhalt in einen neuen Tag schreiben.
	    \begin{lstlisting}[gobble=4,language=Java]
	      $dbf->write_kimb_new( $teil, $inhalt );
	    \end{lstlisting}
	    
	    \begin{tabular}{|lcp{0.7\textwidth}|}
		    \hline
		      \multicolumn{3}{|l|}{ \textbf{Parameter} } \\
		    \hline
		      \$teil & String & \begin{itshape} Tag für den neuen Inhalt \end{itshape} \\
		      \$inhalt & String & \begin{itshape} Neuer Inhalt \end{itshape} \\
		    \hline
		      \multicolumn{3}{|l|}{ \textbf{Rückgabe} } \\
		    \hline
			     & Boolean & \begin{itshape} Erfolgreich? \end{itshape} \\
		    \hline
	    \end{tabular}
	    \begin{flushright} \small \hyperref[tab:methodenuerbersicht]{Zur Methodenübersicht} \end{flushright}
  
  \subsubsection{\$this->write\_kimb\_replace}
  \label{sec:mth_replace}
	    Einen Inhalt in einem vorhanden Tag ersetzen. (Tag darf nur einmal vorhanden sein)
	    \begin{lstlisting}[gobble=4,language=Java]
	      $dbf->write_kimb_replace( $teil, $inhalt );
	    \end{lstlisting}
	    
	    \begin{tabular}{|lcp{0.7\textwidth}|}
		    \hline
		      \multicolumn{3}{|l|}{ \textbf{Parameter} } \\
		    \hline
		      \$teil & String & \begin{itshape} Tag mit dem alten Inhalt \end{itshape} \\
		      \$inhalt & String & \begin{itshape} Neuer Inhalt \end{itshape} \\
		    \hline
		      \multicolumn{3}{|l|}{ \textbf{Rückgabe} } \\
		    \hline
			     & Boolean & \begin{itshape} Erfolgreich? \end{itshape} \\
		    \hline
	    \end{tabular}
	    \begin{flushright} \small \hyperref[tab:methodenuerbersicht]{Zur Methodenübersicht} \end{flushright}
  
  \subsubsection{\$this->write\_kimb\_delete}
  \label{sec:mth_del}
	    Einen Tag löschen. (Tag darf nur einmal vorhanden sein)
	    \begin{lstlisting}[gobble=4,language=Java]
	      $dbf->write_kimb_delete( $teil );
	    \end{lstlisting}
	    
	    \begin{tabular}{|lcp{0.7\textwidth}|}
		    \hline
		      \multicolumn{3}{|l|}{ \textbf{Parameter} } \\
		    \hline
		      \$teil & String & \begin{itshape} Zu löschender Tag \end{itshape} \\
		    \hline
		      \multicolumn{3}{|l|}{ \textbf{Rückgabe} } \\
		    \hline
			     & Boolean & \begin{itshape} Erfolgreich? \end{itshape} \\
		    \hline
	    \end{tabular}
	    \begin{flushright} \small \hyperref[tab:methodenuerbersicht]{Zur Methodenübersicht} \end{flushright}
  
  \subsubsection{\$this->write\_kimb\_one} 
  \label{sec:mth_one_write}
	    Inhalt einfach unter einem Tag ablegen. (Tag wird erstellt, sofern dieser noch nicht vorhanden ist, andernfalls wird der Inhalt ersetzt.)
	    \begin{lstlisting}[gobble=4,language=Java]
	      $dbf->write_kimb_one( $teil, $inhalt );
	    \end{lstlisting}
	    
	    \begin{tabular}{|lcp{0.7\textwidth}|}
		    \hline
		      \multicolumn{3}{|l|}{ \textbf{Parameter} } \\
		    \hline
		      \$teil & String & \begin{itshape} Tag für den Inhalt \end{itshape} \\
		      \$inhalt & String & \begin{itshape} Neuer Inhalt \end{itshape} \\
		    \hline
		      \multicolumn{3}{|l|}{ \textbf{Rückgabe} } \\
		    \hline
			     & Boolean & \begin{itshape} Erfolgreich? \end{itshape} \\
		    \hline
	    \end{tabular}
	    \begin{flushright} \small \hyperref[tab:methodenuerbersicht]{Zur Methodenübersicht} \end{flushright}
  
%
  \subsubsection{\$this->read\_kimb\_search\_teilpl}
  \label{sec:mth_search_tpl}
	    Inhalt unter einem Tag in einer Aufzählung suchen.
	    \begin{lstlisting}[gobble=4,language=Java]
	      $dbf->read_kimb_search_teilpl( $teil, $search );
	    \end{lstlisting}
	    
	    \begin{tabular}{|lcp{0.7\textwidth}|}
		    \hline
		      \multicolumn{3}{|l|}{ \textbf{Parameter} } \\
		    \hline
		      \$teil & String & \begin{itshape} Tag der zu durchsuchenden Inhalte \end{itshape} \\
		      \$search & String & \begin{itshape} Gesuchter Inhalt \end{itshape} \\
		    \hline
		      \multicolumn{3}{|l|}{ \textbf{Rückgabe} } \\
		    \hline
			     & Boolean & \begin{itshape} Gefunden? \end{itshape} \\
		    \hline
	    \end{tabular}
	    \begin{flushright} \small \hyperref[tab:methodenuerbersicht]{Zur Methodenübersicht} \end{flushright}
  
  \subsubsection{\$this->read\_kimb\_all\_teilpl}
  \label{sec:mth_all_tpl}
	    Alle Inhalte unter einem Tag in einer Aufzählung ausgeben.
	    \begin{lstlisting}[gobble=4,language=Java]
	      $dbf->read_kimb_all_teilpl( $teil );
	    \end{lstlisting}
	    
	     \begin{tabular}{|lcp{0.7\textwidth}|}
		    \hline
		      \multicolumn{3}{|l|}{ \textbf{Parameter} } \\
		    \hline
		      \$teil & String & \begin{itshape} Tag der Inhalte \end{itshape} \\
		    \hline
		      \multicolumn{3}{|l|}{ \textbf{Rückgabe} } \\
		    \hline
			     & Array & \begin{itshape} Inhalte \end{itshape} \\
		    \hline
	    \end{tabular}
	    \begin{flushright} \small \hyperref[tab:methodenuerbersicht]{Zur Methodenübersicht} \end{flushright}
  
  \subsubsection{\$this->write\_kimb\_teilpl}
  \label{sec:mth_tpl}
	    Inhalte einer Aufzählung hinzufügen, ändern und löschen.
	    \begin{lstlisting}[gobble=4,language=Java]
	      $dbf->write_kimb_teilpl( $teil, $inhalt, $todo );
	    \end{lstlisting}
	    
	     \begin{tabular}{|lcp{0.7\textwidth}|}
		    \hline
		      \multicolumn{3}{|l|}{ \textbf{Parameter} } \\
		    \hline
		      \$teil & String & \begin{itshape} Tag des Inhalts/ der Aufzählung \end{itshape} \\
		      \$inhalt & String & \begin{itshape} Neuer oder zu löschender Inhalt \end{itshape} \\
		      \$todo & String & \lstinline* add * \begin{itshape} Inhalt neu hinzufügen oder verändern \end{itshape} \\
			     &		& \lstinline* del * \begin{itshape} Inhalt aus der Aufzählung löschen \end{itshape} \\
		    \hline
		      \multicolumn{3}{|l|}{ \textbf{Rückgabe} } \\
		    \hline
			     & Boolean & \begin{itshape} Erfolgreich? \end{itshape} \\
		    \hline
	    \end{tabular}
	    \begin{flushright} \small \hyperref[tab:methodenuerbersicht]{Zur Methodenübersicht} \end{flushright}
  
  \subsubsection{\$this->write\_kimb\_teilpl\_del\_all}
  \label{sec:mth_tpl_del}
	    Alle Inhalte und die gesamte Aufzählung löschen.
	    \begin{lstlisting}[gobble=4,language=Java]
	      $dbf->write_kimb_teilpl_del_all( $teil );
	    \end{lstlisting}
	    
	     \begin{tabular}{|lcp{0.7\textwidth}|}
		    \hline
		      \multicolumn{3}{|l|}{ \textbf{Parameter} } \\
		    \hline
		      \$teil & String & \begin{itshape} Tag der Inhalte/ Aufzählung \end{itshape} \\
		    \hline
		      \multicolumn{3}{|l|}{ \textbf{Rückgabe} } \\
		    \hline
			     & Boolean & \begin{itshape} Erfolgreich? \end{itshape} \\
		    \hline
	    \end{tabular}
	    \begin{flushright} \small \hyperref[tab:methodenuerbersicht]{Zur Methodenübersicht} \end{flushright}
  
%  
  \subsubsection{\$this->read\_kimb\_id}
  \label{sec:mth_id}
	    Inhalte einer Untergruppe einer ID lesen oder alle Untergruppen einer ID lesen.
	    \begin{lstlisting}[gobble=4,language=Java]
	      $dbf->read_kimb_id( $id [, $xxxid = '---all---'] );
	    \end{lstlisting}
	    
	     \begin{tabular}{|lcp{0.7\textwidth}|}
		    \hline
		      \multicolumn{3}{|l|}{ \textbf{Parameter} } \\
		    \hline
		      \$id & Int & \begin{itshape} ID mit Inhalten und Untergruppen \end{itshape} \\
		      \$xxxid & String & \begin{itshape} Name der Untergruppe, deren Inhalt ausgeben werden soll (nicht übergeben => Array mit Inhalten aller Untergruppen)\end{itshape} \\
		    \hline
		      \multicolumn{3}{|l|}{ \textbf{Rückgabe} } \\
		    \hline
			  & String & \begin{itshape} Inhalt der gewählten Untergruppe  \end{itshape} \\
			  & Array & \begin{itshape} Inhalt aller Untergruppen der ID  \end{itshape} \\
			   \multicolumn{3}{|r|}{ abhängig von \$xxxid } \\
		    \hline
	    \end{tabular}
	    \begin{flushright} \small \hyperref[tab:methodenuerbersicht]{Zur Methodenübersicht} \end{flushright}
  
  \subsubsection{\$this->read\_kimb\_all\_xxxid}
  \label{sec:mth_id_all_xx}
	    Alle Untergruppen einer ID ausgeben.
	    \begin{lstlisting}[gobble=4,language=Java]
	      $dbf->read_kimb_all_xxxid( $id );
	    \end{lstlisting}
	    
	     \begin{tabular}{|lcp{0.7\textwidth}|}
		    \hline
		      \multicolumn{3}{|l|}{ \textbf{Parameter} } \\
		    \hline
		      \$id & Int & \begin{itshape} ID, von der man die Untergruppen wissen will \end{itshape} \\
		    \hline
		      \multicolumn{3}{|l|}{ \textbf{Rückgabe} } \\
		    \hline
			  & Array & \begin{itshape} Array mit allen Untergruppen der gewählten ID \end{itshape} \\
		    \hline
	    \end{tabular}
	    \begin{flushright} \small \hyperref[tab:methodenuerbersicht]{Zur Methodenübersicht} \end{flushright}

	    
\subsubsection{\$this->read\_kimb\_id\_all}
 \label{sec:mth_id_all}
	    Alle Inhalte aller IDs in der Datei ausgeben.
	    \begin{lstlisting}[gobble=4,language=Java]
	      $dbf->read_kimb_id_all();
	    \end{lstlisting}
	    
	     \begin{tabular}{|lcp{0.7\textwidth}|}
		    \hline
		      \multicolumn{3}{|l|}{ \textbf{Parameter} } \\
		    \hline
		       \multicolumn{3}{|c|}{ keine } \\
		    \hline
		      \multicolumn{3}{|l|}{ \textbf{Rückgabe} } \\
		    \hline
			  & Array & \begin{itshape} Array mit allen IDs, Untergruppen und Inhalten \end{itshape} \\
			  &  & \lstinline* array( ID =>   * \\
			  &  & \lstinline* 	array( XXXID => Inhalt, ... ), * \\
			  &  & \lstinline*ID => .... ) * \\

		    \hline
	    \end{tabular}
	    \begin{flushright} \small \hyperref[tab:methodenuerbersicht]{Zur Methodenübersicht} \end{flushright}
  
  \subsubsection{\$this->search\_kimb\_id}
  \label{sec:mth_id_search}
	    Suchen nach einem Inhalt in allen Untergruppen einer ID.
	    \begin{lstlisting}[gobble=4,language=Java]
	      $dbf->search_kimb_id( $search , $id ); 
	    \end{lstlisting}
	    
	     \begin{tabular}{|lcp{0.7\textwidth}|}
		    \hline
		      \multicolumn{3}{|l|}{ \textbf{Parameter} } \\
		    \hline
		      \$search & String & \begin{itshape} Gesuchter Inhalt einer Untergruppe \end{itshape} \\
		      \$id & Int & \begin{itshape} ID, deren Untergruppen durchsucht werden \end{itshape} \\
		    \hline
		      \multicolumn{3}{|l|}{ \textbf{Rückgabe} } \\
		    \hline
			  & Boolean/ String & \begin{itshape} Wenn Inhalt gefunden, dann Name der Untergruppe, sonst false \end{itshape} \\
		    \hline
	    \end{tabular}
	    \begin{flushright} \small \hyperref[tab:methodenuerbersicht]{Zur Methodenübersicht} \end{flushright}
  
  \subsubsection{\$this->search\_kimb\_xxxid}
  \label{sec:mth_search_xx}
	    Durchsuchen von allen IDs nach einem Inhalt in einer Untergruppe.
	    \begin{lstlisting}[gobble=4,language=Java]
	$dbf->search_kimb_xxxid( $search, $xxxid ); 
	    \end{lstlisting}
	    
	     \begin{tabular}{|lcp{0.7\textwidth}|}
		    \hline
		      \multicolumn{3}{|l|}{ \textbf{Parameter} } \\
		    \hline
		      \$search & String & \begin{itshape} Gesuchter Inhalt einer Untergruppe \end{itshape} \\
		      \$xxxid & String & \begin{itshape} Untergruppe, welche durchsucht werden soll \end{itshape} \\
		    \hline
		      \multicolumn{3}{|l|}{ \textbf{Rückgabe} } \\
		    \hline
			  & Boolean/ Int & \begin{itshape} Wenn Inhalt gefunden, dann ID, sonst false \end{itshape} \\
		    \hline
	    \end{tabular}
	    \begin{flushright} \small \hyperref[tab:methodenuerbersicht]{Zur Methodenübersicht} \end{flushright}
  
  \subsubsection{\$this->next\_kimb\_id}
  \label{sec:mth_id_next}
	    Herausfinden der nächsten freien ID in der aktuellen KIMBdbf. \\
	    \begin{lstlisting}[gobble=4,language=Java]
	      $dbf->next_kimb_id(); 
	    \end{lstlisting}
	    
	     \begin{tabular}{|lcp{0.7\textwidth}|}
		    \hline
		      \multicolumn{3}{|l|}{ \textbf{Parameter} } \\
		    \hline
		      \multicolumn{3}{|c|}{ keine } \\
		    \hline
		      \multicolumn{3}{|l|}{ \textbf{Rückgabe} } \\
		    \hline
			  & Int & \begin{itshape} Nächste freie ID \end{itshape} \\
		    \hline
	    \end{tabular}
	    \begin{flushright} \small \hyperref[tab:methodenuerbersicht]{Zur Methodenübersicht} \end{flushright}
  
  \subsubsection{\$this->write\_kimb\_id}
  \label{sec:mth_id_write}
	    Schreiben und löschen von Inhalten in IDs und Untergruppen. \\
	    \begin{lstlisting}[gobble=4,language=Java]
     $dbf->write_kimb_id(
       $id, $todo [, $xxxid = '---none---'
       		[, $inhalt = '---none---' [, $oldmode = false ]]]
     ); 
	    \end{lstlisting}
	    
	     \begin{tabular}{|lcp{0.7\textwidth}|}
		    \hline
		      \multicolumn{3}{|l|}{ \textbf{Parameter} } \\
		    \hline
		       \$id & Int & \begin{itshape} ID, in der geschrieben werden soll (wird automatisch erstellt, sollte sie leer sein) \end{itshape} \\
			    & 	 & \begin{itshape} (Bei \lstinline* 0 * wird automatisch eine freie ID  gesucht. [Auto\_Increment]) \end{itshape} \\
			    & 	 & \begin{itshape} (Unter \lstinline* public KIMBdbf::last_written_id * finden Sie die zuletzt veränderte ID.) \end{itshape} \\ 
		       
		       \$todo & String & \lstinline* add * \begin{itshape} Untergruppe neu hinzufügen oder Inhalt verändern \end{itshape} \\
			     &		& \lstinline* del * \begin{itshape} Untergruppe oder ID löschen \end{itshape} \\
			     
		       \$xxxid & String & \begin{itshape} Untergruppe, in der gearbeitet werden soll (nicht übergeben \& \$todo \lstinline* del * => gesamte ID löschen)\end{itshape} \\
		       
		       \$inhalt & String & \begin{itshape} Inhalt, der geschrieben werden soll (nicht übergeben nur sinnvoll wenn ID oder Untergruppe gelöscht werden soll)\end{itshape} \\
		       
		       \$oldmode & Boolean & \begin{itshape} Auto\_Increment deaktivieren, dadurch lässt sich die ID \lstinline* 0 * schreiben.\end{itshape} \\
		    \hline
		      \multicolumn{3}{|l|}{ \textbf{Rückgabe} } \\
		    \hline
			  & Boolean & \begin{itshape} Erfolgreich? \end{itshape} \\
		    \hline
	    \end{tabular}
	    \begin{flushright} \small \hyperref[tab:methodenuerbersicht]{Zur Methodenübersicht} \end{flushright}
  
  \subsubsection{\$this->write\_kimb\_id\_array}
  \label{sec:mth_id_write_arr}
	    Ein Array mit mehreren IDs in einer KIMBdbf ablegen.\\
	    Quasi das schreibende Gegenstück zu \hyperref[sec:mth_id_all]{ \lstinline* $dbf->read_kimb_id_all() * }.
	    \begin{lstlisting}[gobble=4,language=Java]
	$dbf->write_kimb_id_array( $array );
	    \end{lstlisting}
	    
	     \begin{tabular}{|lcp{0.7\textwidth}|}
		    \hline
		      \multicolumn{3}{|l|}{ \textbf{Parameter} } \\
		    \hline
		      \$array & Array & \begin{itshape} Array mit IDs, Untergruppen und Inhalten \end{itshape} \\
			  &  & \lstinline* array( ID =>   * \\
			  &  & \lstinline* 	array( XXXID => Inhalt, ... ), * \\
			  &  & \lstinline*ID => .... ) * \\
			  &  & \begin{itshape} Achtung, die Inhalte der IDs/ Untergruppen werden überschrieben.\end{itshape} \\
		      
		    \hline
		      \multicolumn{3}{|l|}{ \textbf{Rückgabe} } \\
		    \hline
			  & Boolean & \begin{itshape} Erfolgreich? \end{itshape} \\
		    \hline
	    \end{tabular}
	    \begin{flushright} \small \hyperref[tab:methodenuerbersicht]{Zur Methodenübersicht} \end{flushright}

  
%  
  \subsubsection{\$this->delete\_kimb\_file}
  \label{sec:mth_fdel}
	    Die gesamte KIMB-Datei löschen. \\
	    \begin{lstlisting}[gobble=4,language=Java]
	      $dbf->delete_kimb_file();
	    \end{lstlisting}
	    
	     \begin{tabular}{|lcp{0.7\textwidth}|}
		    \hline
		      \multicolumn{3}{|l|}{ \textbf{Parameter} } \\
		    \hline
		      \multicolumn{3}{|c|}{ keine } \\
		    \hline
		      \multicolumn{3}{|l|}{ \textbf{Rückgabe} } \\
		    \hline
			     & Boolean & \begin{itshape} Erfolgreich? \end{itshape} \\
		    \hline
	    \end{tabular}
	    \begin{flushright} \small \hyperref[tab:methodenuerbersicht]{Zur Methodenübersicht} \end{flushright}

  \subsubsection{\$this->show\_kimb\_file}
  \label{sec:mth_fshow}
	    Die gesamte KIMB-Datei, wie sie auf dem Dateisystem liegt, zurückgeben.
	    \begin{lstlisting}[gobble=4,language=Java]
	      $dbf->show_kimb_file();
	    \end{lstlisting}
	    
	     \begin{tabular}{|lcp{0.7\textwidth}|}
		    \hline
		      \multicolumn{3}{|l|}{ \textbf{Parameter} } \\
		    \hline
		      \multicolumn{3}{|c|}{ keine } \\
		    \hline
		      \multicolumn{3}{|l|}{ \textbf{Rückgabe} } \\
		    \hline
			     & String & \begin{itshape} Dateiinhalt \end{itshape} \\
		    \hline
	    \end{tabular}
	    \begin{flushright} \small \hyperref[tab:methodenuerbersicht]{Zur Methodenübersicht} \end{flushright}
  
%
%
%
 \newpage
 \subsection{Funktionen}
 Bei den Funktionen ist nur der Zugriff auf die Datensatzart \glqq Normal\grqq ~möglich. Außerdem ist keine Wahl des \glqq/kimb-data/\grqq ~Pfades möglich. Auch die Verschlüsselung ist immer deaktviert.\\
 Die Funktionen arbeiten technisch gesehen auch mit der KIMBdbf Klasse.
  
  \subsubsection{read\_kimb\_one}
	    Lesen des Inhalts eines Tags. (wenn mehrfach vorhanden, erster Inhalt)
	    \begin{lstlisting}[gobble=4,language=Java]
	      read_kimb_one( $datei, $teil );
	    \end{lstlisting}
	    
	    \begin{tabular}{|lcp{0.7\textwidth}|}
		    \hline
		      \multicolumn{3}{|l|}{ \textbf{Parameter} } \\
		    \hline
		      \$datei & String & \begin{itshape} Name der KIMB-Datei (wie für das Objekt)\end{itshape} \\
		      \$teil & String & \begin{itshape} Tag des Inhalts \end{itshape} \\
		    \hline
		      \multicolumn{3}{|l|}{ \textbf{Rückgabe} } \\
		    \hline
			     & String & \begin{itshape} Inhalt \end{itshape} \\
		    \hline
	    \end{tabular}
	    
  \subsubsection{read\_kimb\_search}
	    Suchen nach einem Tag mit bestimmtem Inhalt, bei mehrfach vorhanden Tags.
	    \begin{lstlisting}[gobble=4,language=Java]
	      read_kimb_search( $datei, $teil, $search );
	    \end{lstlisting}
	    
	    \begin{tabular}{|lcp{0.7\textwidth}|}
		    \hline
		      \multicolumn{3}{|l|}{ \textbf{Parameter} } \\
		    \hline
		      \$datei & String & \begin{itshape} Name der KIMB-Datei (wie für das Objekt)\end{itshape} \\
		      \$teil & String & \begin{itshape} Tag der zu durchsuchenden Inhalte \end{itshape} \\
		      \$search & String & \begin{itshape} Gesuchter Inhalt \end{itshape} \\
		    \hline
		      \multicolumn{3}{|l|}{ \textbf{Rückgabe} } \\
		    \hline
			     & Boolean & \begin{itshape} Gefunden? \end{itshape} \\
		    \hline
	    \end{tabular}
	   
  \subsubsection{write\_kimb\_new}
	    Einen Inhalt in einen neuen Tag schreiben.
	    \begin{lstlisting}[gobble=4,language=Java]
	      write_kimb_new( $datei, $teil, $inhalt );
	    \end{lstlisting}
	    
	    \begin{tabular}{|lcp{0.7\textwidth}|}
		    \hline
		      \multicolumn{3}{|l|}{ \textbf{Parameter} } \\
		    \hline
		      \$datei & String & \begin{itshape} Name der KIMB-Datei (wie für das Objekt)\end{itshape} \\
		      \$teil & String & \begin{itshape} Tag für den neuen Inhalt \end{itshape} \\
		      \$inhalt & String & \begin{itshape} Neuer Inhalt \end{itshape} \\
		    \hline
		      \multicolumn{3}{|l|}{ \textbf{Rückgabe} } \\
		    \hline
			     & Boolean & \begin{itshape} Erfolgreich? \end{itshape} \\
		    \hline
	    \end{tabular}
  \subsubsection{write\_kimb\_replace}
	    Einen Inhalt in einem vorhanden Tag ersetzen. (Tag darf nur einmal vorhanden sein)
	    \begin{lstlisting}[gobble=4,language=Java]
	      write_kimb_replace( $datei, $teil, $inhalt );
	    \end{lstlisting}
	    
	    \begin{tabular}{|lcp{0.7\textwidth}|}
		    \hline
		      \multicolumn{3}{|l|}{ \textbf{Parameter} } \\
		    \hline
		      \$datei & String & \begin{itshape} Name der KIMB-Datei (wie für das Objekt)\end{itshape} \\
		      \$teil & String & \begin{itshape} Tag mit dem alten Inhalt \end{itshape} \\
		      \$inhalt & String & \begin{itshape} Neuer Inhalt \end{itshape} \\
		    \hline
		      \multicolumn{3}{|l|}{ \textbf{Rückgabe} } \\
		    \hline
			     & Boolean & \begin{itshape} Erfolgreich? \end{itshape} \\
		    \hline
	    \end{tabular}
	    
  \subsubsection{write\_kimb\_delete}
	    Einen Tag löschen. (Tag darf nur einmal vorhanden sein)
	    \begin{lstlisting}[gobble=4,language=Java]
	      write_kimb_delete( $datei, $teil );
	    \end{lstlisting}
	    
	    \begin{tabular}{|lcp{0.7\textwidth}|}
		    \hline
		      \multicolumn{3}{|l|}{ \textbf{Parameter} } \\
		    \hline
		      \$datei & String & \begin{itshape} Name der KIMB-Datei (wie für das Objekt)\end{itshape} \\
		      \$teil & String & \begin{itshape} Zu löschender Tag \end{itshape} \\
		    \hline
		      \multicolumn{3}{|l|}{ \textbf{Rückgabe} } \\
		    \hline
			     & Boolean & \begin{itshape} Erfolgreich? \end{itshape} \\
		    \hline
	    \end{tabular}
  
  \subsubsection{delete\_kimb\_datei}
	    Die gesamte KIMB-Datei löschen.
	    \begin{lstlisting}[gobble=4,language=Java]
	      delete_kimb_datei( $datei );
	    \end{lstlisting}
	    
	     \begin{tabular}{|lcp{0.7\textwidth}|}
		    \hline
		      \multicolumn{3}{|l|}{ \textbf{Parameter} } \\
		    \hline
		         \$datei & String & \begin{itshape} Name der KIMB-Datei (wie für das Objekt)\end{itshape} \\
		    \hline
		      \multicolumn{3}{|l|}{ \textbf{Rückgabe} } \\
		    \hline
			     & Boolean & \begin{itshape} Erfolgreich? \end{itshape} \\
		    \hline
	    \end{tabular}
 
  \newpage
  
  \subsection{Überlagerung}
  
  Bei den KIMB-Dateien ist kein vollständiger Schutz vor Überschreiben/ Mehrfachbelegung eingebaut, 
  so kann man mit einer Methode für einen normalen Datensatz auch Aufzählung oder ID-
  Zuordnungen umschreiben. Dies ist kein Fehler, sollte aber jedem Entwickler im 
  Hinterkopf bleiben. Dadurch ermöglicht sich ein einfacher Aufbau eigener Funktionen 
  oder auch das Bearbeiten eines Inhalts ohne die dafür vorgesehene Funktion.\\
  Um Überschreiben zu verhindern bietet sich folgendes an:
  
  \begin{enumerate}
   \item Alle ID's sind Zahlen
   \item Keine Tags fangen mit Zahlen an (Schutz für ID-Zuordnung)
   \item Keine Tags enden mit Zahlen (Schutz für Aufzählungen)
   \item Aufzählungen erhalten einen Präfix, z.B.: „tp-“
   \item Untergruppen bestehen nur aus Buchstaben
   \item Normale Datensätze erhalten einen Präfix, z.B.: „nt-“
  \end{enumerate}
  
  \newpage
  \section{Links}
  
  \begin{description}
    \item[KIMBdbf] \hfill \\
          \href{https://github.com/kimbtech/KIMBdbf}{https://github.com/kimbtech/KIMBdbf}\\
    \item[KIMB-technologies] \hfill \\
	  CMS, Downloader und Co.\\
	  \href{https://kimbtech.github.io/}{https://kimbtech.github.io/}
    \item[GIT] \hfill \\
	  \href{https://bitbucket.org/kimbtech/}{https://bitbucket.org/kimbtech/}\\
	  \href{https://github.com/kimbtech/}{https://gtihub.com/kimbtech/}
    \item[Lizenz] \hfill \\
	\href{http://www.gnu.org/licenses/gpl-3.0.txt}{http://www.gnu.org/licenses/gpl-3.0.txt}
  \end{description}

\end{document}
